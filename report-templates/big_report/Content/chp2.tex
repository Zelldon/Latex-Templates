\chapter{Anderes Kapitel}
	\section{Abbildung}
        \begin{figure}[H]
			\centering
			\includegraphics[scale=0.5]{Content/Logo}
			\caption{Logo}
			\label{fig:logo}
		\end{figure} \noindent
	\section{Tabelle}
		\begin{table}[H]
			\begin{tabular}{| p{3cm} | p{3cm} | p{3cm} | p{5cm} |}
			\hline
			\textbf{Header} 				&		\textbf{Header}				&			\textbf{Header}		&		\textbf{Langer Header}			
			\\ \hline
			Test									&			Test								&			Test							&		Test		
			\\ \hdashline
			\dots								&			\dots							&			\dots						&			\dots
			\\ \hline
			\dots								&			\dots							&			\dots						&			\dots
			\\ \hline
			\end{tabular}
			\caption{Tabelle}
		\end{table}
	\section{Beispiel}
		\subsection{Java}
			\lstinputlisting[language=java,
						numbers=left,
	                 	caption={Java file},
	                 	captionpos=b,
	                 	label=list:java-file]
	                 	{Content/test.java}
		\subsection{Json}
			\lstinputlisting[language=json,
						numbers=left,
	                 	caption={JSON file},
	                 	captionpos=b,
	                 	label=list:json-file]
	                 	{Content/test.json}
		\subsection{Property}
			\lstinputlisting[language=properties,
						numbers=left,
	                 	caption={Property file},
	                 	captionpos=b,
	                 	label=list:property-file]
	                 	{Content/test.properties}
		\subsection{Header}
			\lstinputlisting[language=header,
						numbers=left,
	                 	caption={Headers},
	                 	captionpos=b,
	                 	firstline=3,
	                 	lastline=9,
	                 	label=list:headers]
	                 	{Content/responses.txt}