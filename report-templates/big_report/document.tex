\documentclass{documentclass}
\RequirePackage{documentpackage}
\usepackage{blindtext}
%========================================================
%====================GLOSSARIES============================
%========================================================

%========================================================
%======================ABKZ-VZ============================
%========================================================

%\newacronym{<label>}{<abbrv>}{<full>}

\newacronym{usw}{usw.}{und so weiter}
\newacronym{vgl}{vgl.}{vergleiche}
\newacronym{ebd}{ebd.}{ebenda}
\newacronym{bzw}{bzw.}{beziehungsweise}
\newacronym{engl}{engl.}{englisch}
\newacronym{z.B.}{z.B.}{zum Beispiel}
\newacronym{d.h.}{d.h.}{das hei\ss t}
\newacronym{u.a.}{u.a.}{und andere}
%========================================================
%=======================GLOSSAR============================
%========================================================


\newglossaryentry{ls}
{
  name=\textit{layered system},
  plural= \textit{layered systems},
  description={Ein System das in Schichten eingeteilt ist.
  					  Die untersten Schichten stellen Funktionalit\"aten f\"ur die
  						oberen bzw. \"au\ss ersten Schichten bereit (\cite[vgl.][46]{phd:REST}).}
}

\newglossaryentry{cache}
{
  name=\textit{cache},
  description={Ein \textit{cache} erm\"oglicht das Zwischenspeichern von Informationen f\"ur die Wiederbenutzung (\cite[vgl.][48]{phd:REST}).}
}

\newglossaryentry{RESTful}
{
  name=\textit{RESTful},
  description={Sind alle REST Bedingungen erf"ullt spricht man von RESTful. Siehe zur Erkl\"arung Abschnitt \ref{sec:rest}.}
}

\newglossaryentry{frame}
{
  name=\textit{framework},
  plural= \textit{frameworks},
  description={Englisch f\"ur Rahmen oder Ger\"ust. Wird in dieser Arbeit benutzt um ein Softwareger\"ust zu bezeichnen das 
  						verwendet wird um Software unter bestimmten Aspekten zu entwickeln.
  						Ein framework stellt meist bestimmte Schnittstellen, Architekturmerkmale und Funktionalit\"aten bereit.}
}

\newglossaryentry{ex}
{
  name=\textit{Exception},
  plural=\textit{Exceptions},
  description={Englisch f\"ur Ausnahme oder Ausnahmefall,
  						wird in dieser Arbeit im Zusammenhang mit der Programmiersprache Java verwendet.
  						Java Exceptions beschreiben einen Ausnahme- oder Fehlerfall innerhalb eines Java Programmes.}
}

\newglossaryentry{overhead}
{
  name=\textit{overhead},
  description={Englisch f\"ur Aufwand oder Mehraufwand,
  						wird in dieser Arbeit zur Beschreibung des Mehraufwands bzw. Mehraufwand von Ressourcenkosten verwendet.
  						Ressourcenkosten k\"onnen Bandbreite oder Latenz  bei der \"Ubertragung von Nachrichten sein.}
}


\newglossaryentry{head}
{
  name=\textit{Header},
  plural=\textit{Headers},
  description={Englisch f\"ur Kopfzeile,
  						wird in dieser Arbeit im Zusammenhang von HTTP benutzt.
  						Bei den HTTP-Headers unterscheidet man zwischen Anfrage- und Antwort-Header.
  						Die HTTP-Header gelten als Sticker auf dem HTTP-Envelope, der zwischen Client und Server versendet wird.
  						Die Header gelten als Metadaten und bestehen aus key-value Paaren (\cite[vgl.][6]{Book:REST}).}
}


\newglossaryentry{eb}
{
  name=\textit{entity body},
  plural=\textit{entity bodies},
  description={ Dieser Begriff wird in dieser Arbeit im Zusammenhang von HTTP benutzt.
  						Entity body steht f\"ur das Dokument oder Repr\"asentation innerhalb des HTTP-Evenlope (\cite[vgl.][6]{Book:REST}).}
}

\newglossaryentry{env}
{
  name=\textit{Envelope},
  plural=\textit{Envelopes},
  description={Englisch f\"ur Umschlag oder H\"ulle, wird in dieser Arbeit im Zusammenhang von HTTP benutzt.
  						Der HTTP-Envelope oder Umschlag ist das was von einem Client zum Server gesendet wird und zur\"uck.
  						 Innerhalb dieses Umschlages befindet sich ein entity body oder auch Repr\"asentation die dabei \"ubertragen wird
  						 (\cite[vgl.][5]{Book:REST}).}
}


\newglossaryentry{descr}
{
  name=\textit{Deskriptor},
  plural=\textit{Deskriptor},
  description={Englisch f\"ur Beschreiber, wird in dieser Arbeit im Zusammenhang vom Deployment und web.xml benutzt.
  						Der Deskriptor oder Deployment-Deskriptor bezeichnet etwas, was den Entwicklungsprozess bzw. die Aufstellung
  						eines Webservices beschreibt.}
}


\newglossaryentry{mar}
{
  name=\textit{marshaller},
  description={Englisch f\"ur Einweiser, wird in dieser Arbeit im Zusammenhang mit der Programmiersprache Java und der
  						 JAXB Spezifikation verwendet.
  						 Der marshaller erm\"oglicht es die Daten/Informationen aus einem Java-Objekt in ein XML oder JSON Format 		
  						 zu konvertieren. Dieses Vorgehen wird als marshalling bezeichnet.}
}


\newglossaryentry{unmar}
{
  name=\textit{unmarshaller},
  description={Dieser Begriff wird in dieser Arbeit im Zusammenhang mit der Programmiersprache Java und der
  						 JAXB Spezifikation verwendet. Der unmarshaller erm\"oglicht die Umkehroperation des marshallings,
  						 bezeichnet als das unmarshalling.
  						 Er konvertiert die Daten aus dem XML oder JSON Format zur\"uck in ein Java-Objekt.}
}


\newglossaryentry{pfile}
{
  name=\textit{property file},
  description={Property englisch f\"ur Eigenschaft. 
  						Der Begriff property file wird in dieser Arbeit im Zusammenhang mit der Programmiersprache Java verwendet.
  						Das property file hat normalerweise die Endung ''properties''.
  						Innerhalb dieses file werden Eigenschaften bzw. Informationen, als key-value Paare, gespeichert. Diese Informationen
  						k\"onnen das Verhalten des zugeh\"origen Java Programms beeinflussen.}
}

\newglossaryentry{kv}
{
  name=\textit{key-value},
  description={Key englisch f\"ur Schl\"ussel und value englisch f\"ur Wert beschreiben ein Schl\"ussel-Wert Paar.
  						Der Schl\"ussel erm\"oglicht es den Wert eindeutig zu identifizieren.
  						Solche key-value Paare werden z.B. im JSON und Form-Encoded Format oder innerhalb eines property files verwendet.
  						Der Schl\"ussel und der dazugeh\"orige Wert wird meist mit einem Zeichen getrennt.
  						Das Zeichen kann z.B. $=$, $:$ oder anderes sein.}
}

\newglossaryentry{entity}
{
  name=\textit{entity},
  plural= \textit{entities},
  description={Englisch f\"ur Etwas oder einen Gegenstand, wird in dieser Arbeit im Zusammenhang mit der Programmiersprache Java 
  						und ORM verwendet.
  						Entity beschreibt in dieser Arbeit eine Klasse oder ein Objekt, das einen Eintrag aus einer Tabelle einer Datenbank repr\"asentiert.
  						Z.B. die Entity-Klasse User repr\"asentiert einen Eintrag aus der User-Tabelle.}
}
%========================================================
%===================BACHELOR-ARBEIT===================
%========================================================
\begin{document}
%========================================================
%========================TITLEPAGE==========================
%========================================================
\tpage
%========================================================
%========================DANKSAGUNG========================
%========================================================
\dank
\pagestyle{headings} % ermoeglicht das ausgeben von kapiteln als header
\pagenumbering{roman}

%========================================================
%========================TOC=============================
%========================================================
\tableofcontents
\cleardoublepage
%========================================================
%=========================FIGURES==========================
%========================================================
%optional for hyperref:
\phantomsection
\addcontentsline{toc}{chapter}{Abbildungsverzeichnis}
\listoffigures 
\cleardoublepage
%========================================================
%=========================TABLES===========================
%========================================================
%optional for hyperref:
\phantomsection
\addcontentsline{toc}{chapter}{Tabellenverzeichnis}
\listoftables
\cleardoublepage
%========================================================
%=========================EXAMPLES=========================
%========================================================
%optional for hyperref:
\phantomsection
\addcontentsline{toc}{chapter}{Beispielverzeichnis}
\lstlistoflistings
\cleardoublepage
\newpage
%========================================================
%==========================CONTENT========================
%========================================================
\setcounter{page}{1}
\pagenumbering{arabic}

\blinddocument
\chapter{Test-Kapitel}

	\section{Abschnitt Test}
		\blindtext
	\section{weiterer Abschnitt}
		\blindtext
		\subsection{Unterabschnitt}
			\subsubsection{Unterunterabschnitt}
			
\chapter{Anderes Kapitel}
	\section{Abbildung}
        \begin{figure}[H]
			\centering
			\includegraphics[scale=0.5]{Content/Logo}
			\caption{Logo}
			\label{fig:logo}
		\end{figure} \noindent
	\section{Tabelle}
		\begin{table}[H]
			\begin{tabular}{| p{3cm} | p{3cm} | p{3cm} | p{5cm} |}
			\hline
			\textbf{Header} 				&		\textbf{Header}				&			\textbf{Header}		&		\textbf{Langer Header}			
			\\ \hline
			Test									&			Test								&			Test							&		Test		
			\\ \hdashline
			\dots								&			\dots							&			\dots						&			\dots
			\\ \hline
			\dots								&			\dots							&			\dots						&			\dots
			\\ \hline
			\end{tabular}
			\caption{Tabelle}
		\end{table}
	\section{Beispiel}
		\subsection{Java}
			\lstinputlisting[language=java,
						numbers=left,
	                 	caption={Java file},
	                 	captionpos=b,
	                 	label=list:java-file]
	                 	{Content/test.java}
		\subsection{Json}
			\lstinputlisting[language=json,
						numbers=left,
	                 	caption={JSON file},
	                 	captionpos=b,
	                 	label=list:json-file]
	                 	{Content/test.json}
		\subsection{Property}
			\lstinputlisting[language=properties,
						numbers=left,
	                 	caption={Property file},
	                 	captionpos=b,
	                 	label=list:property-file]
	                 	{Content/test.properties}
		\subsection{Header}
			\lstinputlisting[language=header,
						numbers=left,
	                 	caption={Headers},
	                 	captionpos=b,
	                 	firstline=3,
	                 	lastline=9,
	                 	label=list:headers]
	                 	{Content/responses.txt}

\glsaddall
\printglossary[type=acronym,title=Abk\"urzungsverzeichnis,toctitle=Abk\"urzungsverzeichnis] 
\printglossary

\setcounter{chapter}{0}
\printbibliography[heading=bibintoc]
\cleardoublepage
\newpage
\setcounter{page}{1}
%\pagestyle{plain} % ermoeglicht das loeschen des headers
\pagenumbering{Alph}
\appendix
\chapter{Anhang}
\cleardoublepage
\newpage
\section{Anhang - Sec 1}
\cleardoublepage
\newpage
\section{Anhang - Sec 2}

%========================================================
%========================ERKLAERUNG========================
%========================================================
\erklaerung
\end{document}